\documentclass[14pt]{extarticle}

\usepackage{mathtools}
\usepackage{tikz}
\usepackage{listings}
\usepackage{amsmath}
\usepackage{graphicx}
\usepackage[landscape]{geometry}


\begin{document}

\section{Ricerca}
\subsection{Criteri di valutazione delle strategie}
\begin{enumerate}
    \item \textbf{Completezza}\\
            Garanzia di trovare una soluzione, se esiste
    \item \textbf{Ottimalità}\\
            Garanzia di trovare una soluzione ottima (a costo minimo)
    \item \textbf{Complessità Temporale}\\
            Quanto tempo occorre per trovare una soluzione
    \item \textbf{Complessità Spaziale}\\
            Quanta memoria occorre per effettuare la ricerca
\end{enumerate}
\subsubsection{Valutazioe della Complessità (Temporale e Spaziale)}
Si usa Tempo e Spazio \textbf{parametrici}. Calcolati in termini di \textbf{numero di nodi creati o visitati}.
\subsection{Strategie di Ricerca}
Ci sono due tipologie di approcci (riguardanti la conoscenza, quanta e quale)
\begin{enumerate}
    \item \textbf{Approcci Blind}\\
        usano esclusivamente la struttura del problema per cercare (e trovare) una soluzione.
    \item \textbf{Approcci Informati}\\
        usano la struttura del problema + ulteriore conoscenza per guidare la ricerca
        \begin{enumerate}
            \item Monoagente
            \item Multiagente (giochi)
            \item Problemi di Assegnamento (\textbf{CSP})
        \end{enumerate}
\end{enumerate}
\subsubsection{RICERCA IN AMPIEZZA}
\begin{itemize}
    \item La ricerca espande il nodo radice, poi tutti i suoi successori, poi tutti i discendenti di secondo livello, ecc.
    \item Si realizza gestendo la frontiera come una coda FIFO.
    \item Tutti i nodi sulla frontiera e tutti i loro antenati vanno tenuti in memoria per poter ricostruire la soluzione quando si trova il nodo obiettivo (Complessità Spaziale)
\end{itemize}
\paragraph{Valutazione}
\subparagraph{Completezza}
Se esiste un nodo obiettivo a una profondità \textbf{finita $d$} esso verrà
trovato \textbf{a patto che} il \textbf{fattore di ramificazione} (cioè il numero di figli che un nodo può avere) $b$ sia finito
\subparagraph{Ottimalità}
La soluzione trovata è ottima \textbf{solo se} il costo del cammino è
\textbf{una funzione monotona crescente della profondità} (es. tutte le azioni hanno lo stesso costo, quindi \textit{il costo} dipende solo dalla profondità dell'albero)
\subparagraph{Complessità temporale}
di tipo \textbf{esponenziale} $O(b^{d+1})$\\
Quanto tempo occorre per trovare una soluzione col crescere dello spazio di ricerca
\subparagraph{Complessità spaziale}
di tipo \textbf{esponenziale} $O(b^{d+1})$\\
Quanta memoria occorre per effettuare la ricerca col crescere dello spazio di ricerca

\subsubsection{RICERCA A COSTO UNIFORME}
\begin{itemize}
    \item Quando i costi dei passi \textbf{non sono tutti identici}:
        \begin{itemize}
            \item ogni nodo ha associato il costo del cammino con cui è stato raggiunto
            \item la frontiera è mantenuta ordinata
            \item a ogni iterazione \textbf{espande il nodo appartenente a un cammino di costo minimo}
        \end{itemize}
    \item Quando i costi sono tutti \textbf{uguali} diventa la ricerca in \textbf{ampiezza}.
    \item \textbf{ATTENZIONE}
    \begin{itemize}
        \item COSTO $\neq$ NUMERO DEI PASSI\\il numero dei passi non conta, conta solo il costo dei cammini
        \item quando trova il nodo obiettivo \textbf{NON SI FERMA SUBITO}, prima controlla se vi sono cammini aperti di costo inferire e, nel caso, prova ad espanderli.
    \end{itemize}
\end{itemize}
\subsubsection{Ricerca in profondità (senza BackTracking)}
\subsubsection{Iterative Deepening}
\subsubsection{Ricerca Bidirezionale}

\end{document}